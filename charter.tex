\documentclass[
11pt, % The default document font size, options: 10pt, 11pt, 12pt
%codirector, % Uncomment to add a codirector to the title page
]{charter} 


% El títulos de la memoria, se usa en la carátula y se puede usar el cualquier lugar del documento con el comando \ttitle
\titulo{Título del proyecto} 

% Nombre del posgrado, se usa en la carátula y se puede usar el cualquier lugar del documento con el comando \degreename
%\posgrado{Carrera de Especialización en Sistemas Embebidos} 
%\posgrado{Carrera de Especialización en Internet de las Cosas} 
\posgrado{Carrera de Especialización en Inteligencia Artificial}
%\posgrado{Maestría en Sistemas Embebidos} 
%\posgrado{Maestría en Internet de las cosas}

% Tu nombre, se puede usar el cualquier lugar del documento con el comando \authorname
% IMPORTANTE: no omitir titulaciones ni tildación en los nombres, también se recomienda escribir los nombres completos (tal cual los tienen en su documento)
\autor{Título y Nombre del autor}

% El nombre del director y co-director, se puede usar el cualquier lugar del documento con el comando \supname y \cosupname y \pertesupname y \pertecosupname
\director{Título y Nombre del director}
\pertenenciaDirector{pertenencia} 
\codirector{} % para que aparezca en la portada se debe descomentar la opción codirector en los parámetros de documentclass
\pertenenciaCoDirector{FIUBA}

% Nombre del cliente, quien va a aprobar los resultados del proyecto, se puede usar con el comando \clientename y \empclientename
\cliente{Nombre del cliente}
\empresaCliente{Empresa del cliente}
 
\fechaINICIO{30 de abril de 2023}		%Fecha de inicio de la cursada de GdP \fechaInicioName
\fechaFINALPlan{18 de junio de 2023} 	%Fecha de final de cursada de GdP
\fechaFINALTrabajo{15 de mayo de 2024}	%Fecha de defensa pública del trabajo final


\begin{document}

\maketitle
\thispagestyle{empty}
\pagebreak


\thispagestyle{empty}
{\setlength{\parskip}{0pt}
\tableofcontents{}
}
\pagebreak


\section*{Registros de cambios}
\label{sec:registro}


\begin{table}[ht]
\label{tab:registro}
\centering
\begin{tabularx}{\linewidth}{@{}|c|X|c|@{}}
\hline
\rowcolor[HTML]{C0C0C0} 
Revisión & \multicolumn{1}{c|}{\cellcolor[HTML]{C0C0C0}Detalles de los cambios realizados} & Fecha      \\ \hline
0      & Creación del documento                                 &\fechaInicioName \\ \hline
%1      & Se completa hasta el punto 5 inclusive                & {día} de {mes} de 202X \\ \hline
%2      & Se completa hasta el punto 9 inclusive
%		  Se puede agregar algo más \newline
%		  En distintas líneas \newline
%		  Así                                                    & {día} de {mes} de 202X \\ \hline
%3      & Se completa hasta el punto 12 inclusive                & {día} de {mes} de 202X \\ \hline
%4      & Se completa el plan	                                 & {día} de {mes} de 202X \\ \hline

% Si hay más correcciones pasada la versión 4 también se deben especificar acá

\end{tabularx}
\end{table}

\pagebreak



\section*{Acta de constitución del proyecto}
\label{sec:acta}

\begin{flushright}
Buenos Aires, \fechaInicioName
\end{flushright}

\vspace{2cm}

Por medio de la presente se acuerda con el \authorname\hspace{1px} que su Trabajo Final de la \degreename\hspace{1px} se titulará ``\ttitle'' y consistirá en \textcolor{red}{la implementación de un prototipo de un sistema de control de temperatura de una caldera industrial}. El trabajo tendrá un presupuesto preliminar estimado de \textcolor{red}{600} horas y un costo estimado de \textcolor{red}{\$ XXX}, con fecha de inicio el \fechaInicioName\hspace{1px} y fecha de presentación pública el \fechaFinalName.

Se adjunta a esta acta la planificación inicial.

\vfill

% Esta parte se construye sola con la información que hayan cargado en el preámbulo del documento y no debe modificarla
\begin{table}[ht]
\centering
\begin{tabular}{ccc}
\begin{tabular}[c]{@{}c@{}}Dr. Ing. Ariel Lutenberg \\ Director posgrado FIUBA\end{tabular} & \hspace{2cm} & \begin{tabular}[c]{@{}c@{}}\clientename \\ \empclientename \end{tabular} \vspace{2.5cm} \\ 
\multicolumn{3}{c}{\begin{tabular}[c]{@{}c@{}} \supname \\ Director del Trabajo Final\end{tabular}} \vspace{2.5cm} \\
\end{tabular}
\end{table}




\section{1. Descripción técnica-conceptual del proyecto a realizar}
\label{sec:descripcion}

\begin{consigna}{red} % ELIMINAR \begin{consigna}{red} y \end{consigna}{red} en las secciones que vayan completando para cada entrega parcial.
El objetivo es que el lector, en una o dos páginas, exponga de qué se trata el proyecto y cuáles son sus desafíos, cuál es la motivación para realizarlo y su importancia.

Se debe introducir el contexto del proyecto, el estado del arte en la temática, describir la propuesta de valor, cuál es el problema que atiende y cuál es la solución que se propone. Se debe dar una descripción funcional de la solución que incluya un diagrama en bloques.

Puede ser útil incluir en esta sección la respuesta a alguna de estas preguntas:

\begin{itemize}
	\item ¿Cuál es el contexto del proyecto, es un emprendimiento personal, un proyecto para una empresa, es parte del programa de vinculación con empresas del posgrado?
	\item ¿Existen o aplican condiciones especiales al proyecto, financiamiento de algún programa público o privado, acuerdos de confidencialidad, acuerdos sobre la propiedad intelectual de los entregables u otros?
	\item ¿Cómo se compara la solución propuesta con el estado del arte en el campo de aplicación? ¿En qué aspectos destaca?
	\item ¿Ayuda a la explicación si se incluye un lienzo Canvas del Modelo de Negocio?
	\item ¿En qué estado del ciclo de vida está la solución que se propone?
	\item ¿Cuáles son las características del cliente (el adoptante de los entregables del proyecto) qué valora, qué necesita?
	\item ¿Por dónde pasa la innovación?
\end{itemize}

La descripción técnica-conceptual \textbf{debe incluir al menos un diagrama en bloques del sistema} y descripción funcional de la solución propuesta.

Las figuras se deben mencionar en el texto ANTES de que aparezcan con una frase como la siguiente: ``En la figura \ref{fig:diagBloques} se presenta el diagrama en bloques del sistema. Se observa que...''.  La regla es que las figuras nunca pueden ir antes de ser mencionadas en el texto, porque sino el lector no entiende por qué de pronto aparece una figura.

\begin{figure}[htpb]
\centering 
\includegraphics[width=.65\textwidth]{./Figuras/diagBloques.png}
\caption{Diagrama en bloques del sistema.}
\label{fig:diagBloques}
\end{figure}

\vspace{25px}

El tamaño del texto en TODAS las figuras debe ser adecuado \textbf{para que NO pase lo que ocurre en la figura \ref{fig:diagBloques}}, donde el lector debe esforzarse para poder leer el texto. 

Los colores usados en el diagrama deben ser adecuados, tal que ayuden a comprender mejor el diagrama. Se recomienda evitar colores primarios (como rojo, verde o cyan) y usar la gama de colores pastel.

\end{consigna} % ELIMINAR \begin{consigna}{red} y \end{consigna}{red} en las secciones que vayan completando para cada entrega parcial.

\section{2. Identificación y análisis de los interesados}
\label{sec:interesados}

\begin{consigna}{red} % ELIMINAR \begin{consigna}{red} y \end{consigna}{red} en las secciones que vayan completando para cada entrega parcial.
\textbf{Nota importante:} borrar esto y todas las consignas en color rojo antes de entregar este documento). Esto se hace eliminando el par de comandos que forman el bloque consigna, \verb!\begin{consigna}{red}! y \verb!\end{consigna}{red}! del código. 
 
Es inusual que una misma persona esté en más de un rol, incluso en proyectos chicos. Si se considera que una persona cumple dos o más roles, entonces \textbf{solo dejarla en el rol más importante}. 

Por ejemplo, si una persona es Cliente pero también colabora u orienta, dejarla solo como Cliente. Si una persona es el Responsable, \textbf{no debe ser colocado también como miembro del equipo}.


\begin{table}[ht]
%\caption{Identificación de los interesados}
%\label{tab:interesados}
\begin{tabularx}{\linewidth}{@{}|l|X|X|l|@{}}
\hline
\rowcolor[HTML]{C0C0C0} 
Rol           & Nombre y Apellido & Organización 	& Puesto 	\\ \hline
Auspiciante   &                   &              	&        	\\ \hline
Cliente       & \clientename      &\empclientename	&        	\\ \hline
Impulsor      &                   &              	&        	\\ \hline
Responsable   & \authorname       & FIUBA        	& Alumno 	\\ \hline
Colaboradores &                   &              	&        	\\ \hline
Orientador    & \supname	      & \pertesupname 	& Director del Trabajo Final \\ \hline
Equipo        & miembro1 \newline 
				miembro2          &              	&        	\\ \hline
Opositores    &                   &              	&        	\\ \hline
Usuario final &                   &              	&        	\\ \hline
\end{tabularx}
\end{table}

El Director suele ser uno de los orientadores.

No dejar celdas vacías; si no hay nada que poner en una celda colocar un signo ``-''.

No dejar filas vacías; si no hay nada que poner en una fila entonces eliminarla.

Es deseable listar a continuación las principales características de cada interesado.
 
Por ejemplo:
\begin{itemize}
	\item Orientador: la Dra. Ing. María Gómez es experta en la temática y va a ayudar con la definición de los requerimientos y el desarrollo del firmware del embebido.
	\item Auspiciante: es riguroso y exigente con la rendición de gastos. Tener mucho cuidado con esto.
	\item Equipo: Juan Perez, suele pedir licencia porque tiene un familiar con una enfermedad. Planificar considerando esto.
\end{itemize}

\end{consigna} % ELIMINAR \begin{consigna}{red} y \end{consigna}{red} en las secciones que vayan completando para cada entrega parcial.


\section{3. Propósito del proyecto}
\label{sec:proposito}

\begin{consigna}{red} % ELIMINAR \begin{consigna}{red} y \end{consigna}{red} en las secciones que vayan completando para cada entrega parcial.

¿Por qué se hace el proyecto? ¿Qué se quiere lograr? 

Se recomienda que sea solo un párrafo que continúe con la idea de la frase ``el propósito de este proyecto es...'' (omitir la frase, ya que está en el título de la sección).
\end{consigna}

\section{4. Alcance del proyecto}
\label{sec:alcance}

\begin{consigna}{red}
¿Qué se incluye y que no se incluye en este proyecto?

Se refiere al trabajo que se va a hacer para entregar el producto o resultado especificado. 

Explicitar todo lo quede comprendido dentro del alcance del proyecto. Por ejemplo:

El proyecto incluye:
\begin{itemize}
	\item Ítem 1.
	\item Ítem 2.
		\begin{itemize}
		\item Subítem 1.
		\item Subítem 2.
		\item ...
		\end{itemize}
	\item ...
	
\end{itemize}

Explicitar además todo lo que no quede incluido (``El presente proyecto no incluye...'')

\end{consigna} % ELIMINAR \begin{consigna}{red} y \end{consigna}{red} en las secciones que vayan completando para cada entrega parcial.


\section{5. Supuestos del proyecto}
\label{sec:supuestos}

\begin{consigna}{red} % ELIMINAR \begin{consigna}{red} y \end{consigna}{red} en las secciones que vayan completando para cada entrega parcial.
``Para el desarrollo del presente proyecto se supone que: ...''

\begin{itemize}
	\item Supuesto 1.
	\item Supuesto 2.
	\item ...
\end{itemize}

Por ejemplo, se podrían incluir supuestos respecto a disponibilidad de tiempo y recursos humanos y materiales, sobre la factibilidad técnica de distintos aspectos del proyecto, sobre otras cuestiones que sean necesarias para el éxito del proyecto como condiciones macroeconómicas o reglamentarias.

\end{consigna} % ELIMINAR \begin{consigna}{red} y \end{consigna}{red} en las secciones que vayan completando para cada entrega parcial.

\section{6. Product Backlog}
\label{sec:backlog}

El Product Backlog debe organizarse en cuatro \textbf{\textit{\'{e}picas}} fundamentales del proyecto. Cada \'{e}pica debe contener al menos dos historias de usuario que describan funcionalidades clave.

El Product Backlog debe permitir interpretar cómo será el proyecto y su funcionalidad. Se deben indicar claramente las prioridades entre las historias de usuario y si hay alguna opcional.

Las historias de usuario deben ser breves, claras y medibles, expresando el rol, la necesidad y el propósito de cada funcionalidad. También deben tener una prioridad definida para facilitar la planificación de los sprints.

Cada historia de usuario debe incluir una ponderación en \textit{Story Points}, un número entero que representa el tama\~no relativo de la historia. El criterio para calcular los Story Points debe indicarse explícitamente.

Las historias deben seguir el formato: ``\textit{Como [rol], quiero [tal cosa] para [tal otra cosa]}''.

Las \'{e}picas deben estructurarse de la siguiente forma:

\begin{itemize}
  \item \textbf{\'{E}pica 1}
    \begin{itemize}
      \item HU1
      \item HU2
    \end{itemize}
  \item \textbf{\'{E}pica 2}
    \begin{itemize}
      \item HU3
      \item HU4
    \end{itemize}
  \item \textbf{\'{E}pica 3}
    \begin{itemize}
      \item HU5
      \item HU6
    \end{itemize}
  \item \textbf{\'{E}pica 4}
    \begin{itemize}
      \item HU7
      \item HU8
    \end{itemize}
\end{itemize}

\textbf{Reglas para definir historias de usuario:}
\begin{itemize}
  \item Ser concisas y claras.
  \item Expresarlas en términos cuantificables y medibles.
  \item No dejar margen para interpretaciones ambiguas.
  \item Indicar claramente su prioridad y si son opcionales.
  \item Considerar regulaciones y normas vigentes.
\end{itemize}

\section{7. Criterios de aceptación de historias de usuario}
\label{sec:criteriosAceptacion}

Los criterios de aceptación deben establecerse para cada historia de usuario, asegurando que se cumplan las condiciones necesarias para que la funcionalidad sea validada correctamente.

Cada historia debe tener criterios medibles, específicos y verificables. Deben permitir validar que se cumple con las necesidades del usuario.

Se estructuran de forma análoga a las \'{e}picas del backlog:

\begin{itemize}
  \item \textbf{\'{E}pica 1}
    \begin{itemize}
      \item Criterios de aceptación HU1
      \item Criterios de aceptación HU2
    \end{itemize}
  \item \textbf{\'{E}pica 2}
    \begin{itemize}
      \item Criterios de aceptación HU3
      \item Criterios de aceptación HU4
    \end{itemize}
  \item \textbf{\'{E}pica 3}
    \begin{itemize}
      \item Criterios de aceptación HU5
      \item Criterios de aceptación HU6
    \end{itemize}
  \item \textbf{\'{E}pica 4}
    \begin{itemize}
      \item Criterios de aceptación HU7
      \item Criterios de aceptación HU8
    \end{itemize}
\end{itemize}

\textbf{Reglas para definir criterios de aceptación:}
\begin{itemize}
  \item Medibles y verificables.
  \item Especificar cuándo una historia se considera completada.
  \item Incluir condiciones específicas.
  \item No ambiguos.
  \item Probables de testear funcional o técnicamente.
  \item Mínimo 3 criterios por HU.
\end{itemize}

\section{8. Fases de CRISP-DM}

\begin{enumerate}
  \item \textbf{Comprensión del negocio:} objetivo, valor agregado de IA, métricas de éxito.
  \item \textbf{Comprensión de los datos:} tipo, origen, cantidad, calidad.
  \item \textbf{Preparación de los datos:} características clave, transformaciones necesarias.
  \item \textbf{Modelado:} tipo de problema, algoritmos posibles.
  \item \textbf{Evaluación del modelo:} métricas de rendimiento.
  \item \textbf{Despliegue del modelo (opcional):} tipo de despliegue y herramientas.
\end{enumerate}

\section{9. Desglose del trabajo en tareas}
\label{sec:wbs}

A partir de cada HU, descomponer en tareas concretas, técnicas y medibles:

\begin{itemize}
  \item Duración estimada: entre 2 y 8 h. Evitar tareas genéricas.
  \item Si una tarea excede 8 h, dividirla.
  \item Indicar prioridad relativa (Alta, Media, Baja).
\end{itemize}

\begin{table}[htpb]
\centering
\begin{tabularx}{\linewidth}{@{}|X|X|c|c|@{}}
\hline
\rowcolor[HTML]{C0C0C0}
Historia de usuario & Tarea técnica & Estimación & Prioridad \\ \hline
HU1 & Tarea 1 HU1 & 6 h & Alta \\ \hline
HU1 & Tarea 2 HU1 & 8 h & Alta \\ \hline
HU2 & Tarea 1 HU2 & 5 h & Media \\ \hline
HU2 & Tarea 2 HU2 & 6 h & Alta \\ \hline
... & ... & ... & ... \\ \hline
\end{tabularx}
\end{table}

\section{10. Diagrama de Gantt}
\label{sec:gantt}

Representar visualmente la planificación temporal de:
\begin{itemize}
  \item Tareas técnicas derivadas de HU.
  \item Tareas no técnicas: planificación, escritura, defensa, etc.
\end{itemize}

\textbf{Requisitos:}
\begin{itemize}
  \item Eje vertical: tareas.
  \item Eje horizontal: semanas o fechas.
  \item Colores distintos para HU vs. generales.
  \item Coincidir fechas con acta constitutiva.
\end{itemize}

\section{11. Planificación de Sprints}

Organizar tareas en sprints de 1 a 3 semanas.
\begin{itemize}
  \item Relacionar sprints con HU y tareas.
  \item Incluir horas estimadas, responsable y \% completado.
  \item Planificar entre 10 y 12 sprints.
\end{itemize}

\begin{table}[htpb]
\centering
\begin{tabularx}{\linewidth}{@{}|X|X|c|X|c|@{}}
\hline
\rowcolor[HTML]{C0C0C0}
Sprint & HU o fase & Tarea & Horas / SP & \% Completado \\ \hline
Sprint 0 & Planificación & Definir alcance y cronograma & 10 h & 100\% \\ \hline
Sprint 1 & HU1 & Tarea 1 HU1 & 6 h / 3 SP & 0\% \\ \hline
... & ... & ... & ... & ... \\ \hline
\end{tabularx}
\end{table}

\section{12. Normativa y cumplimiento de datos (gobernanza)}

Analizar:
\begin{itemize}
  \item Si los datos están regulados por GDPR, Ley 25.326, HIPAA u otras.
  \item Si requieren consentimiento explícito.
  \item Si tienen licencia pública, privada o restricción de uso.
\end{itemize}

Esto permite evaluar la viabilidad legal y ética del proyecto.



\section{13. Gestión de riesgos}
\label{sec:riesgos}

\begin{consigna}{red}
a) Identificación de los riesgos (al menos cinco) y estimación de sus consecuencias:
 
Riesgo 1: detallar el riesgo (riesgo es algo que si ocurre altera los planes previstos de forma negativa)
\begin{itemize}
	\item Severidad (S): mientras más severo, más alto es el número (usar números del 1 al 10).\\
	Justificar el motivo por el cual se asigna determinado número de severidad (S).
	\item Probabilidad de ocurrencia (O): mientras más probable, más alto es el número (usar del 1 al 10).\\
	Justificar el motivo por el cual se asigna determinado número de (O). 
\end{itemize}   

Riesgo 2:
\begin{itemize}
	\item Severidad (S): X.\\
	Justificación...
	\item Ocurrencia (O): Y.\\
	Justificación...
\end{itemize}

Riesgo 3:
\begin{itemize}
	\item Severidad (S):  X.\\
	Justificación...
	\item Ocurrencia (O): Y.\\
	Justificación...
\end{itemize}


b) Tabla de gestión de riesgos:      (El RPN se calcula como RPN=SxO)

\begin{table}[htpb]
\centering
\begin{tabularx}{\linewidth}{@{}|X|c|c|c|c|c|c|@{}}
\hline
\rowcolor[HTML]{C0C0C0} 
Riesgo & S & O & RPN & S* & O* & RPN* \\ \hline
       &   &   &     &    &    &      \\ \hline
       &   &   &     &    &    &      \\ \hline
       &   &   &     &    &    &      \\ \hline
       &   &   &     &    &    &      \\ \hline
       &   &   &     &    &    &      \\ \hline
\end{tabularx}%
\end{table}

Criterio adoptado: 

Se tomarán medidas de mitigación en los riesgos cuyos números de RPN sean mayores a...

Nota: los valores marcados con (*) en la tabla corresponden luego de haber aplicado la mitigación.

c) Plan de mitigación de los riesgos que originalmente excedían el RPN máximo establecido:
 
Riesgo 1: plan de mitigación (si por el RPN fuera necesario elaborar un plan de mitigación).
  Nueva asignación de S y O, con su respectiva justificación:
  \begin{itemize}
	\item Severidad (S*): mientras más severo, más alto es el número (usar números del 1 al 10).
          Justificar el motivo por el cual se asigna determinado número de severidad (S).
	\item Probabilidad de ocurrencia (O*): mientras más probable, más alto es el número (usar del 1 al 10).
          Justificar el motivo por el cual se asigna determinado número de (O).
	\end{itemize}

Riesgo 2: plan de mitigación (si por el RPN fuera necesario elaborar un plan de mitigación).
 
Riesgo 3: plan de mitigación (si por el RPN fuera necesario elaborar un plan de mitigación).

\end{consigna}


\section{14. Gestión de la calidad}
\label{sec:calidad}

\begin{consigna}{red}
Elija al menos diez requerimientos que a su criterio sean los más importantes/críticos/que aportan más valor y para cada uno de ellos indique las acciones de verificación y validación que permitan asegurar su cumplimiento.

\begin{itemize} 
\item Req \#1: copiar acá el requerimiento con su correspondiente número.

\begin{itemize}
	\item Verificación para confirmar si se cumplió con lo requerido antes de mostrar el sistema al cliente. Detallar.
	\item Validación con el cliente para confirmar que está de acuerdo en que se cumplió con lo requerido. Detallar. 
\end{itemize}

\end{itemize}

Tener en cuenta que en este contexto se pueden mencionar simulaciones, cálculos, revisión de hojas de datos, consulta con expertos, mediciones, etc.  

Las acciones de verificación suelen considerar al entregable como ``caja blanca'', es decir se conoce en profundidad su funcionamiento interno.  

En cambio, las acciones de validación suelen considerar al entregable como ``caja negra'', es decir, que no se conocen los detalles de su funcionamiento interno.

\end{consigna}

\section{15. Procesos de cierre}    
\label{sec:cierre}

\begin{consigna}{red}
Establecer las pautas de trabajo para realizar una reunión final de evaluación del proyecto, tal que contemple las siguientes actividades:

\begin{itemize}
	\item Pautas de trabajo que se seguirán para analizar si se respetó el Plan de Proyecto original:\\
	 - Indicar quién se ocupará de hacer esto y cuál será el procedimiento a aplicar. 
	\item Identificación de las técnicas y procedimientos útiles e inútiles que se emplearon, los problemas que surgieron y cómo se solucionaron:\\
	 - Indicar quién se ocupará de hacer esto y cuál será el procedimiento para dejar registro.
	\item Indicar quién organizará el acto de agradecimiento a todos los interesados, y en especial al equipo de trabajo y colaboradores:\\
	  - Indicar esto y quién financiará los gastos correspondientes.
\end{itemize}

\end{consigna}

\end{document}